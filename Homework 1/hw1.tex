%!TEX encoding = UTF-8 Unicode

\documentclass{essay}
\usepackage[margin=1in]{geometry}
\usepackage{mathtools}
\usepackage{minted}

\begin{document}
\sffamily

\setsmallheader{Problem Set 1}{ECEV 35600}{Ansel George}
\setbodyheader{Ansel George}{Problem Set 1}

\begin{essaystyle}

\textbf{1.} The observed counts of genotypes of humans at the MN blood group
locus in two samples are:

\begin{tabular}{l | r r r}
Genotype             & MM  & NM & NN \\ \hline
New Guinea Highlands & 2   & 32 & 269  \\
Guatemalan Indians   & 112 & 74 & 17\\
\end{tabular}

\textbf{(a)} Estimate the allele frequencies in each population.

New Guinea Highlands:

\begin{align}
N_{NG} &= 2 + 32 + 269 = 303
\end{align}
\begin{align}
  P_{M,NG} &= \frac{N_{MM,NG}}{N_{NG}} + \frac{1}{2}\frac{N_{MN,NG}}{N_{NG}} \\
  &= \frac{2}{303} + \frac{1}{2}\frac{32}{303} = 0.05940594
\end{align}
\begin{align}
  P_{N,NG} &= \frac{N_{NN,NG}}{N_{NG}} + \frac{1}{2}\frac{N_{MN,NG}}{N_{NG}} \\
  &= \frac{269}{303} + \frac{1}{2}\frac{32}{303} = 0.9405941
\end{align}

Guatemalan Highlands:

\begin{align}
N_{NG} &= 112+74+17  = 203
\end{align}
\begin{align}
  P_{M,G} &= \frac{N_{MM,G}}{N_{G}} + \frac{1}{2}\frac{N_{MN,G}}{N_{G}} \\
  &= \frac{112}{203} + \frac{1}{2}\frac{74}{203} = 0.7339901
\end{align}
\begin{align}
  P_{N,G} &= \frac{N_{NN,G}}{N_{G}} + \frac{1}{2}\frac{N_{MN,G}}{N_{NG}} \\
  &= \frac{17}{203} + \frac{1}{2}\frac{74}{203} = 0.2660099
\end{align}


\textbf{(b)} Does each population appear to be in Hardy-Weinberg equilibrium?

For populations to be in Hardy-Weinberg equilibrium, the genotype frequencies
must match the products of the respective genotype frequencies, i.e. $P_{AA} =
P_A^2$ and $P_{AB} = 2P_A P_B$.

New Guinea Highlands:

\begin{align}
  E_{HW,NG}\big[P_{MM,NG}\big] &= P_{M,NG}^2 \\
  &= 0.05940594^2 *303 = 1.069307\\
  E_{HW,NG}\big[P_{MN,NG}\big] &= P_{M,NG}P_{N,NG}N_{NG} \\
  &= 2* 0.05940594*0.9405941*303 = 33.8614\\
  E_{HW,NG}\big[P_{NN,NG}\big] &= P_{N,NG}^2 \\
  &= 0.9405941^2 *303 = 268.0693
\end{align}

Guatemalan Highlands:

\begin{align}
  E_{HW,G}\big[P_{MM,G}\big] &= P_{M,G}^2 \\
  &= 0.7339901^2 *203 = 109.3645 \\
  E_{HW,G}\big[P_{MN,G}\big] &= P_{M,G}P_{N,G}N_{G} \\
  &= 2*0.7339901*0.2660099*203 = 79.27095 \\
  E_{HW,G}\big[P_{NN,G}\big] &= P_{N,G}^2 \\
  &= 0.2660099^2 *203 =21.44066
\end{align}

To assess whether the populations likely deviate from HW equilibrium, compute
the $\chi^2$ statistic:

New Guinea Highlands:

\begin{align}
  \chi^2 &= \frac{(2-1.069307)^2}{1.069307} + \frac{(32-33.8614)^2}{33.8614} + \frac{(269-268.0693)^2}{268.0693} \\
  &= 0.9156021
\end{align}

Given 2 degrees of freedom and the low $\chi^2$ value, there is not much reason
to believe the New Guinea Highlands population deviates substantially from HW
equilibrium (p-value = 0.6327).

Guatemalan Highlands:

\begin{align}
  \chi^2 &= \frac{(112-109.3645)^2}{109.3645} + \frac{(74-79.27095)^2}{79.27095} + \frac{(17-14.36453)^2}{14.36453} \\
  &= 0.89752
\end{align}

Given 2 degrees of freedom and the low $\chi^2$ value, there is not much reason
to believe the New Guinea Highlands population deviates substantially from HW
equilibrium, either (p-value = 0.6384).


\textbf{(c)} Suppose the two samples were combined, and analyzed. Would the
combined sample appear to come from a population at Hardy-Weinberg equilibrium?
(Calculate the observed and expected genotype frequencies (or counts)).

The combined population has the following parameters:

\begin{align}
N_{Total} &= N_{NG} + N_G = 303 + 203 = 506
\end{align}
\begin{align}
  P_{M,Total} &= \frac{N_{MM,NG} + N_{MM,G}}{N_{Total}} + \frac{1}{2}\frac{N_{MN,NG}+N_{NM,G}}{N_{Total}} \\
  &= \frac{2+112}{506} + \frac{1}{2}\frac{32+74}{506} = 0.3300395
\end{align}
\begin{align}
  P_{N,Total} &= \frac{N_{NN,NG} + N_{NN,G}}{N_{Total}} + \frac{1}{2}\frac{N_{MN,NG}+N_{NM,G}}{N_{Total}} \\
  &= \frac{269+17}{506} + \frac{1}{2}\frac{32+74}{506} = 0.6699605
\end{align}

The HW expectations are:

\begin{align}
  E_{HW,Total}\big[P_{MM,Total}\big] &= P_{M,Total}^2 \\
  &= 0.3300395^2 *506 = 55.11659 \\
  E_{HW,Total}\big[P_{MN,Total}\big] &= P_{M,Total}P_{N,Total}N_{G} \\
  &= 2*0.3300395*0.6699605*506 = 223.7668 \\
  E_{HW,Total}\big[P_{NN,Total}\big] &= P_{N,Total}^2 \\
  &= 0.6699605^2 *506 = 227.1166
\end{align}

The $\chi^2$ statistic for the combined population is:

\begin{align}
  \chi^2 &= \frac{(114-55.11659)^2}{55.11659} + \frac{(106-223.7668)^2}{223.7668} + \frac{(286 - 227.1166)^2}{227.1166} \\
  &= 140.1539
\end{align}

When combined, the population appears to deviate substantially from HW
equilibrium (p-value < 2.2e-16).


\textbf{2.} A sample of Peromyscus polionotus were genotyped at the esterase-2
locus using protein electrophoresis. Three alleles, denoted a, b and c were
observed. The observed counts of the six genotypes were:

\begin{tabular}{l | r r r r r r}
Genotype & aa & bb & cc & ab & ac & bc \\
Count    & 2  & 53 & 5  & 13 & 3  & 21 \\
\end{tabular}

\textbf{(a)} Estimate the allele frequencies of each of the alleles. Also, calculate the
number of each genotype expected in the sample under Hardy-Weinberg
equilibrium. 

\begin{align}
P_a &= \frac{2 + \frac{1}{2} 13 + \frac{1}{2} 3}{2 + 53 + 5 + 13 + 3 + 21} = \frac{10}{97} = 0.1030928 \\
P_b &= \frac{53 + \frac{1}{2} 13 + \frac{1}{2} 21}{97} = \frac{70}{97} = 0.7216495 \\
P_c &= \frac{5 + \frac{1}{2} 3 + \frac{1}{2} 21}{97} = \frac{17}{97} = 0.1752577
\end{align}

Under HW equilibrium:

\begin{align}
E_{aa} &= N * P_a^2 = 97 * 0.1030928^2 = 1.030928 \\
E_{bb} &= N * P_b^2 = 97 * 0.7216495^2 = 50.51547 \\
E_{cc} &= N * P_c^2 = 97 * 0.1752577^2 = 2.97938 \\
E_{ab} &= N * 2P_a P_b = 97 * 2 * 0.1030928 * 0.7216495 = 14.43299 \\
E_{ac} &= N * 2P_a P_c = 97 * 2 * 0.1030928 * 0.1752577 = 3.505155 \\
E_{bc} &= N * 2P_b P_c = 97 * 2 * 0.7216495 * 0.1752577 = 24.53608
\end{align}


\textbf{3.} A small island is colonized by a monoecious (hermaphroditic) plant.
The population size remains constant at 40 individuals. Consider an autosomal
locus that initially has heterozygosity of 0.4. 

\textbf{(a)} What is the expected heterozygosity after 55 generations?

Assuming that only self-fertilization occurs, the heterozygosity will decrease
by 50 percent each generation because only 50 \% of selfing heterozygotes'
offspring are heterozygous themselves at a given locus.

$Het_{t=55} = .4 * .5^{55} = 1.110223 * 10^{-17} \approx 0$

\textbf{(b)} If the population size had been 400, what would the expected heterozygosity
be after 55 generations?

Heterozygosity is the frequency of heterozygotes in the population, and is
independent of population tize. The heterozytote frequencies at 55 generations
will remain the same for a population of 40 or 400 selfing individuals.

$Het_{t=55} = .4 * .5^{55} = 1.110223 * 10^{-17} \approx 0$


\textbf{4.} Suppose five random copies of a gene were sequenced. Assume the
gene is 1200 base pairs long, and that all sites were monomorphic (unvarying)
in this sample except for six nucleotide sites. The sequences at the variable
sites were: 

seq 1: AAAAAG seq 2: CTAGAG seq 3: AACATG seq 4: AACATG seq 5: CTAGAT 

\textbf{(a)} Calculate the nucleotide diversity using the average number of pairwise
differences between the sampled copies. 

\begin{minted}[linenos,numbersep=.5em]{r}
countPairwiseDifferences <- function(sequence1, sequence2) {
  str1array = strsplit(sequence1, "")[[1]]
  str2array = strsplit(sequence2, "")[[1]]
  count = 0
  for (i in 1:length(str1array)) {
    if (str1array[i] != str2array[i]) {
      count = count + 1
    }
  }
  return(count)
}

sequences = c("AAAAAG", "CTAGAG", "AACATG", "AACATG", "CTAGAT")

nSites = 1200
totalPairwiseDifferences <- 0
sequenceCombinations <- combn(sequences, 2)
for (i in 1:dim(sequenceCombinations)[2]) { 
  totalPairwiseDifferences = totalPairwiseDifferences + 
    countPairwiseDifferences(sequenceCombinations[1,i], sequenceCombinations[2,i])
}

nucleotideDiversity = totalPairwiseDifferences * 1/5 * 1/5 / nSites # 1/5 <- allele frequency

[1] 0.001133333
\end{minted}

Nucleotide diversity across all 1200 base pairs is $\pi = 0.002833333$.

\textbf{(b)} Consider the following data, (obtained from the above data by simply
scrambling the alleles within columns). 

seq 1: ATAATG seq 2: AAAGAG seq 3: CACAAG seq 4: ATCATG seq 5: CAAGAT 

By construction, the allele frequencies are the same as in the previous
example, so the nucleotide diversity must be the same. Verify that the average
number of pairwise differences is the same as before. 

\begin{minted}[linenos,numbersep=.5em]{r}
sequences = c("ATAATG", "AAAGAG", "CACAAG", "ATCATG", "CAAGAT")

nSites = 1200
totalPairwiseDifferences <- 0
sequenceCombinations <- combn(sequences, 2)
for (i in 1:dim(sequenceCombinations)[2]) { 
  totalPairwiseDifferences = totalPairwiseDifferences + 
    countPairwiseDifferences(sequenceCombinations[1,i], sequenceCombinations[2,i])
}

nucleotideDiversity = totalPairwiseDifferences * 1/5 * 1/5 / nSites # 1/5 <- allele frequency

[1] 0.001133333
\end{minted}

The diversity measured from the permuted data is identical to that from the original.

\textbf{(c)} Calculate Watterson’s estimator of $\theta$ per base pair. (The estimate
based on the number of segregating sites.) Compare this estimate to the
estimate of $\theta$ per base pair based on nucleotide diversity. Why might
they differ?

The calculation for Watterson's estimator is:

\begin{align}
  \hat\theta_w &= \frac{K}{a_n}
\end{align}

In this case, $K = 6$ and $a_n$ is the harmonic sum of 1200:

\begin{align}
  a_n &= \sum^{n-1}_{i=1}\frac{1}{i} \\
  &= \frac{1}{1} + \frac{1}{2} + \frac{1}{3} + \ldots + \frac{1}{1198} + \frac{1}{1199} \\
  &\approx 7.666042
\end{align}

Therefore, $\hat\theta_w = \frac{6}{7.666042} \frac{1}{1200} = 0.0006522271$.

Watterson's estimator is less than the nucleotide diversity in this case. They
differ because they are computed differently.

Nucleotide diversity relies on a frequency-scaled count of the pairwise
differences along different alleles, where SNPs at each site can be considered
independently from the others. The Watterson's estimator instead makes the
assumption that there is no recombination along segregating sites, meaning
unique sequences are generated by combinations of SNPs that are inherited as a
unit. The metric is concerned with how K segregating sites are distributed
along a tract of DNA\@ given some sample of the population. The harmonic sum in
its denominator is in a sense a correction factor for the likelihood of
sampling to fail to capture rare variants in the population.

The difference in the measured $\pi$ and $\theta$ could be due to unspecified
population structure, such as variation in population size over time, or other
violations of model assumptions.


\textbf{5.} Consider a population surveyed at two loci that are 1cM apart, with
two alleles at each locus. (1cM means that the recombination fraction is 0.01).
Suppose a large sample was genotyped and the frequencies of the four possible
haplotypes were: 

$freq(A_1 – B_1) = 0.4$

$freq(A_1 – B_2) = 0.2$

$freq(A_2 – B_1) = 0.1$

$freq(A_2 – B_2) = 0.3$

\textbf{(a)} What is the frequency of $A_1$ and $B_1$?

$freq_{A_1} = .4 + .2 = .6$ \\
$freq_{B_1} = .4 + .1 = .5$ \\

\textbf{(b)} What is the linkage disequilibrium (D)

$D = freq_{A_1 B_1} - freq_{A_1}freq_{B_1} = .4 - .6*.5 = .1$

\textbf{(c)} Assuming no mutation, selection or drift, what allele frequencies, gamete
frequencies and linkage disequilibrium do you expect after 100 generations of
random mating?

$D_{t} = (1-r)^t D_0$

After 100 generations with $D_0 = .1$ and $r = .01$:

$D_{t=100} = (1-.01)^{100} *.1 = 0.03660323$

Given that there is no mutation, selection, or drift, the allele frequencies
will remain same within the population, so the genotype frequencies will
become:

\begin{align}
freq_{A_1 B_1} &= D + freq_{A_1} freq_{B_1} \\
  &= 0.03660323 + .6*.5 = 0.33660323 \\
freq_{a_1 b_1} &= D + freq_{a_1} freq_{b_1} \\
  &= 0.03660323 + .4*.5 = 0.23660323 \\
freq_{A_1 b_1} &= freq_{A_1} freq_{b_1} - D \\
  &= .6*.5 - 0.03660323 = 0.2633968 \\
freq_{a_1 B_1} &= freq_{a_1} freq_{B_1} - D \\
  &= .4*.5 - 0.03660323 = 0.1633968
\end{align}


\textbf{6.} An anthropologist found that, in a population with a complicated
system of kin marriages, 17 of 800 people are homozygous for isozyme allele AS.
Later the marriage system broke down and mating became random. The frequency of
these homozygotes then dropped to 0.01. What was the average inbreeding
coefficient under the old system?

The frequency of AS homozygotes being .01 with random mating and .02125 under
the old system. Assuming the underlying allele frequencies remained constant
over time, $p_{AS} = \sqrt{p_{AS/AS}} = \sqrt{.01} = .1$.

The imbreeding coefficient F is given by:

\begin{align}
  F &= \frac{p_{AS,AS} - p_{AS}^2}{p_{AS} - p_{AS}^2} \\
    &= \frac{\frac{17}{800} - .1^2}{.1 - .1^2} \\
    &= 0.125
\end{align}


\textbf{7.} The formula $H = \frac{4N_e u}{4N_e u + 1}$ measures absolute
heterozygosity, whereas the formula $H_t = H_0(1 - \frac{1}{2N_e})^t$ measures
only relative heterozygosity. What assumption, other than selective
neutrality, is required to make the first formula correct?

The first formula represents the equilbrium heterozygosity given random
mutations and drift. It makes the assumpation that there are infinitely many
mutable sites, meaning that each mutation is unique.


\textbf{8.} Which are most closely related ($F_{ij}$) for X-linked loci: brothers,
sisters, or brother and sister?

Sisters inherit one X chromosome from their father, so they will be on average
100\% identical for one chromosome and 50\% identical for the one they inherit
from their mother due to recombination.

$F_{ij} = .5 * .5 + .5 * 1 = .75$

Brothers will only share on average 50\% of loci on the single X chromosome
they inherit from their mother.

$F_{ij} = 1 * .5 = .5$

Brothers and sisters will share 50\% of loci on the X inherited maternally, and
all loci daughters inherit from their father will not be shared by their
brothers because they inherited a Y chromosome instead.

For brother and sister:

$F_{ij} = 1 * .5 = .5$

For sister and brother:

$F_{ij} = .5 * .5 + 0 * .5 = .25$

Therefore, for X-linked loci, sisters are most closely related to one another
than other sets of siblings.

\end{essaystyle}
\end{document}
